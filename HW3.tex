% Options for packages loaded elsewhere
\PassOptionsToPackage{unicode}{hyperref}
\PassOptionsToPackage{hyphens}{url}
%
\documentclass[
]{article}
\usepackage{amsmath,amssymb}
\usepackage{iftex}
\ifPDFTeX
  \usepackage[T1]{fontenc}
  \usepackage[utf8]{inputenc}
  \usepackage{textcomp} % provide euro and other symbols
\else % if luatex or xetex
  \usepackage{unicode-math} % this also loads fontspec
  \defaultfontfeatures{Scale=MatchLowercase}
  \defaultfontfeatures[\rmfamily]{Ligatures=TeX,Scale=1}
\fi
\usepackage{lmodern}
\ifPDFTeX\else
  % xetex/luatex font selection
\fi
% Use upquote if available, for straight quotes in verbatim environments
\IfFileExists{upquote.sty}{\usepackage{upquote}}{}
\IfFileExists{microtype.sty}{% use microtype if available
  \usepackage[]{microtype}
  \UseMicrotypeSet[protrusion]{basicmath} % disable protrusion for tt fonts
}{}
\makeatletter
\@ifundefined{KOMAClassName}{% if non-KOMA class
  \IfFileExists{parskip.sty}{%
    \usepackage{parskip}
  }{% else
    \setlength{\parindent}{0pt}
    \setlength{\parskip}{6pt plus 2pt minus 1pt}}
}{% if KOMA class
  \KOMAoptions{parskip=half}}
\makeatother
\usepackage{xcolor}
\usepackage[margin=1in]{geometry}
\usepackage{color}
\usepackage{fancyvrb}
\newcommand{\VerbBar}{|}
\newcommand{\VERB}{\Verb[commandchars=\\\{\}]}
\DefineVerbatimEnvironment{Highlighting}{Verbatim}{commandchars=\\\{\}}
% Add ',fontsize=\small' for more characters per line
\usepackage{framed}
\definecolor{shadecolor}{RGB}{248,248,248}
\newenvironment{Shaded}{\begin{snugshade}}{\end{snugshade}}
\newcommand{\AlertTok}[1]{\textcolor[rgb]{0.94,0.16,0.16}{#1}}
\newcommand{\AnnotationTok}[1]{\textcolor[rgb]{0.56,0.35,0.01}{\textbf{\textit{#1}}}}
\newcommand{\AttributeTok}[1]{\textcolor[rgb]{0.13,0.29,0.53}{#1}}
\newcommand{\BaseNTok}[1]{\textcolor[rgb]{0.00,0.00,0.81}{#1}}
\newcommand{\BuiltInTok}[1]{#1}
\newcommand{\CharTok}[1]{\textcolor[rgb]{0.31,0.60,0.02}{#1}}
\newcommand{\CommentTok}[1]{\textcolor[rgb]{0.56,0.35,0.01}{\textit{#1}}}
\newcommand{\CommentVarTok}[1]{\textcolor[rgb]{0.56,0.35,0.01}{\textbf{\textit{#1}}}}
\newcommand{\ConstantTok}[1]{\textcolor[rgb]{0.56,0.35,0.01}{#1}}
\newcommand{\ControlFlowTok}[1]{\textcolor[rgb]{0.13,0.29,0.53}{\textbf{#1}}}
\newcommand{\DataTypeTok}[1]{\textcolor[rgb]{0.13,0.29,0.53}{#1}}
\newcommand{\DecValTok}[1]{\textcolor[rgb]{0.00,0.00,0.81}{#1}}
\newcommand{\DocumentationTok}[1]{\textcolor[rgb]{0.56,0.35,0.01}{\textbf{\textit{#1}}}}
\newcommand{\ErrorTok}[1]{\textcolor[rgb]{0.64,0.00,0.00}{\textbf{#1}}}
\newcommand{\ExtensionTok}[1]{#1}
\newcommand{\FloatTok}[1]{\textcolor[rgb]{0.00,0.00,0.81}{#1}}
\newcommand{\FunctionTok}[1]{\textcolor[rgb]{0.13,0.29,0.53}{\textbf{#1}}}
\newcommand{\ImportTok}[1]{#1}
\newcommand{\InformationTok}[1]{\textcolor[rgb]{0.56,0.35,0.01}{\textbf{\textit{#1}}}}
\newcommand{\KeywordTok}[1]{\textcolor[rgb]{0.13,0.29,0.53}{\textbf{#1}}}
\newcommand{\NormalTok}[1]{#1}
\newcommand{\OperatorTok}[1]{\textcolor[rgb]{0.81,0.36,0.00}{\textbf{#1}}}
\newcommand{\OtherTok}[1]{\textcolor[rgb]{0.56,0.35,0.01}{#1}}
\newcommand{\PreprocessorTok}[1]{\textcolor[rgb]{0.56,0.35,0.01}{\textit{#1}}}
\newcommand{\RegionMarkerTok}[1]{#1}
\newcommand{\SpecialCharTok}[1]{\textcolor[rgb]{0.81,0.36,0.00}{\textbf{#1}}}
\newcommand{\SpecialStringTok}[1]{\textcolor[rgb]{0.31,0.60,0.02}{#1}}
\newcommand{\StringTok}[1]{\textcolor[rgb]{0.31,0.60,0.02}{#1}}
\newcommand{\VariableTok}[1]{\textcolor[rgb]{0.00,0.00,0.00}{#1}}
\newcommand{\VerbatimStringTok}[1]{\textcolor[rgb]{0.31,0.60,0.02}{#1}}
\newcommand{\WarningTok}[1]{\textcolor[rgb]{0.56,0.35,0.01}{\textbf{\textit{#1}}}}
\usepackage{graphicx}
\makeatletter
\def\maxwidth{\ifdim\Gin@nat@width>\linewidth\linewidth\else\Gin@nat@width\fi}
\def\maxheight{\ifdim\Gin@nat@height>\textheight\textheight\else\Gin@nat@height\fi}
\makeatother
% Scale images if necessary, so that they will not overflow the page
% margins by default, and it is still possible to overwrite the defaults
% using explicit options in \includegraphics[width, height, ...]{}
\setkeys{Gin}{width=\maxwidth,height=\maxheight,keepaspectratio}
% Set default figure placement to htbp
\makeatletter
\def\fps@figure{htbp}
\makeatother
\setlength{\emergencystretch}{3em} % prevent overfull lines
\providecommand{\tightlist}{%
  \setlength{\itemsep}{0pt}\setlength{\parskip}{0pt}}
\setcounter{secnumdepth}{-\maxdimen} % remove section numbering
\ifLuaTeX
  \usepackage{selnolig}  % disable illegal ligatures
\fi
\IfFileExists{bookmark.sty}{\usepackage{bookmark}}{\usepackage{hyperref}}
\IfFileExists{xurl.sty}{\usepackage{xurl}}{} % add URL line breaks if available
\urlstyle{same}
\hypersetup{
  pdftitle={HW3},
  pdfauthor={Zhuodiao Kuang},
  hidelinks,
  pdfcreator={LaTeX via pandoc}}

\title{HW3}
\author{Zhuodiao Kuang}
\date{2023-10-09}

\begin{document}
\maketitle

\begin{Shaded}
\begin{Highlighting}[]
\FunctionTok{library}\NormalTok{(MASS)}
\FunctionTok{library}\NormalTok{(dplyr)}
\end{Highlighting}
\end{Shaded}

\hypertarget{problem-1}{%
\section{Problem 1}\label{problem-1}}

Some medical professionals claim that the average weight of American
women is 171 pounds. The column lwt holds the mother's weight (in
pounds) at last menstrual period, i.e.~her pre-pregnancy weight. Use
this column for the following questions.

\hypertarget{a-construct-a-95-confidence-interval-of-true-mean-weight-of-american-women}{%
\paragraph{a) Construct a 95\% confidence interval of true mean weight
of American
women}\label{a-construct-a-95-confidence-interval-of-true-mean-weight-of-american-women}}

\[\frac{\bar{X}-\mu}{s/\sqrt{n}} \sim t(n-1)\]

So, the interval is

\[\bar{X}-\frac{s}{\sqrt{n}}t_{0.975} \le \mu \le \bar{X}+\frac{s}{\sqrt{n}}t_{0.975}\]

\begin{Shaded}
\begin{Highlighting}[]
\NormalTok{wt }\OtherTok{\textless{}{-}}\NormalTok{ birthwt }\SpecialCharTok{|\textgreater{}} \FunctionTok{pull}\NormalTok{(lwt)}
\NormalTok{n }\OtherTok{\textless{}{-}} \FunctionTok{length}\NormalTok{(wt)}
\NormalTok{df }\OtherTok{=}\NormalTok{ n}\DecValTok{{-}1}
\NormalTok{t }\OtherTok{\textless{}{-}} \FunctionTok{qt}\NormalTok{(}\FloatTok{0.975}\NormalTok{,}\AttributeTok{df =}\NormalTok{ df)}
\NormalTok{lq }\OtherTok{\textless{}{-}} \FunctionTok{mean}\NormalTok{(wt) }\SpecialCharTok{{-}} \FunctionTok{sd}\NormalTok{(wt)}\SpecialCharTok{/}\FunctionTok{sqrt}\NormalTok{(n)}\SpecialCharTok{*}\NormalTok{t}
\NormalTok{uq }\OtherTok{\textless{}{-}} \FunctionTok{mean}\NormalTok{(wt) }\SpecialCharTok{+} \FunctionTok{sd}\NormalTok{(wt)}\SpecialCharTok{/}\FunctionTok{sqrt}\NormalTok{(n)}\SpecialCharTok{*}\NormalTok{t}
\FunctionTok{print}\NormalTok{(}\FunctionTok{c}\NormalTok{(lq,uq))}
\end{Highlighting}
\end{Shaded}

\begin{verbatim}
[1] 125.4270 134.2027
\end{verbatim}

So, the 95\% confidence interval of true mean weight of American women
is \([125.4270 ,134.2027]\).

\hypertarget{b-interpret-the-confidence-interval.}{%
\paragraph{b) Interpret the confidence
interval.}\label{b-interpret-the-confidence-interval.}}

We are 95\% confident that the average weight of all American women in
the population is between 125.4270 and 134.2027 pounds. This means that
if we repeated the same sampling procedure many times and calculated a
confidence interval for each sample, about 95\% of these intervals would
contain the true population mean weight of American women.

\hypertarget{c-comment-on-the-validity-of-the-statement-above-some-medical-professionals-claim-that-the-average-weight-of-american-women-is-171-pounds.-in-other-words-what-can-we-say-about-this-statement-given-our-confidence-interval-from-part-a}{%
\paragraph{c) Comment on the validity of the statement above (``Some
medical professionals claim that the average weight of American women is
171 pounds''). In other words, what can we say about this statement
given our confidence interval from part
a?}\label{c-comment-on-the-validity-of-the-statement-above-some-medical-professionals-claim-that-the-average-weight-of-american-women-is-171-pounds.-in-other-words-what-can-we-say-about-this-statement-given-our-confidence-interval-from-part-a}}

Given our confidence interval of {[}125.4270 ,134.2027{]} for the true
mean weight of American women, we can say that the statement that the
average weight of American women is 171 pounds is very unlikely to be
true. This is because 171 pounds is far outside the range of plausible
values for the population mean weight of American women based on our
sample data. If the statement were true, it would mean that our sample
was not representative of the population or that there was a large
sampling error or bias in our data collection process. Therefore, we
have strong evidence to reject the claim that the average weight of
American women is 171 pounds.

\hypertarget{problem-2}{%
\section{Problem 2}\label{problem-2}}

In this data set, we have a variable (smoke) indicating the smoking
status of the mothers during pregnancy. Some doctors believe that
smoking status is related to weight. Using the columns smoke and lwt,
test this claim. (Note: a value of 1 indicates the mother is in the
``smoking'' group.)

\hypertarget{a-test-for-the-equality-of-variances-between-the-two-groups.-use-a-5-significance-level.}{%
\paragraph{a) Test for the equality of variances between the two groups.
(Use a 5\% significance
level.)}\label{a-test-for-the-equality-of-variances-between-the-two-groups.-use-a-5-significance-level.}}

Testing the hypothesis:
\[H_0 : \sigma_1^2 = \sigma_2^2   vs   H_1 : \sigma_1^2 \neq \sigma_2^2\]

\[F =\frac{s_1^2}{s_2^2}\sim F_{n_1-1,n_2-1}   under H_0\]

\begin{Shaded}
\begin{Highlighting}[]
\NormalTok{wt1 }\OtherTok{\textless{}{-}}\NormalTok{ birthwt }\SpecialCharTok{|\textgreater{}} \FunctionTok{filter}\NormalTok{(smoke }\SpecialCharTok{==} \DecValTok{1}\NormalTok{) }\SpecialCharTok{|\textgreater{}} \FunctionTok{pull}\NormalTok{(lwt)}
\NormalTok{wt0 }\OtherTok{\textless{}{-}}\NormalTok{ birthwt }\SpecialCharTok{|\textgreater{}} \FunctionTok{filter}\NormalTok{(smoke }\SpecialCharTok{==} \DecValTok{0}\NormalTok{) }\SpecialCharTok{|\textgreater{}} \FunctionTok{pull}\NormalTok{(lwt)}
\NormalTok{n1 }\OtherTok{\textless{}{-}} \FunctionTok{length}\NormalTok{(wt1)}
\NormalTok{n0 }\OtherTok{\textless{}{-}} \FunctionTok{length}\NormalTok{(wt0)}
\NormalTok{s1 }\OtherTok{\textless{}{-}} \FunctionTok{var}\NormalTok{(wt1)}
\NormalTok{s0 }\OtherTok{\textless{}{-}} \FunctionTok{var}\NormalTok{(wt0)}
\NormalTok{F }\OtherTok{=}\NormalTok{ s1}\SpecialCharTok{/}\NormalTok{s0}

\NormalTok{lq }\OtherTok{\textless{}{-}}\FunctionTok{qf}\NormalTok{(}\FloatTok{0.025}\NormalTok{,n1}\DecValTok{{-}1}\NormalTok{,n0}\DecValTok{{-}1}\NormalTok{)}
\NormalTok{uq }\OtherTok{\textless{}{-}}\FunctionTok{qf}\NormalTok{(}\FloatTok{0.975}\NormalTok{,n1}\DecValTok{{-}1}\NormalTok{,n0}\DecValTok{{-}1}\NormalTok{)}

\NormalTok{F;lq;uq }
\end{Highlighting}
\end{Shaded}

\begin{verbatim}
[1] 1.412636
\end{verbatim}

\begin{verbatim}
[1] 0.6518345
\end{verbatim}

\begin{verbatim}
[1] 1.50466
\end{verbatim}

\hypertarget{b-given-your-answer-from-part-a-what-kind-of-hypothesis-test-will-you-perform}{%
\paragraph{b) Given your answer from part a, what kind of hypothesis
test will you
perform?}\label{b-given-your-answer-from-part-a-what-kind-of-hypothesis-test-will-you-perform}}

Given that two population variances are equal, we can test the
hypothesis that both groups have the same mean(two-sided).

Testing the hypothesis:
\[H_0 : \mu_1 = \mu_2   vs   H_1 : \mu_1 \neq \mu_2\]

\hypertarget{c-conduct-your-chosen-hypothesis-test-from-part-b-at-the-10-significance-level.-what-is-your-decision-regarding-the-null-interpret-this-result-in-the-context-of-the-problem.}{%
\paragraph{c) Conduct your chosen hypothesis test from part b at the
10\% significance level. What is your decision regarding the null?
Interpret this result in the context of the
problem.}\label{c-conduct-your-chosen-hypothesis-test-from-part-b-at-the-10-significance-level.-what-is-your-decision-regarding-the-null-interpret-this-result-in-the-context-of-the-problem.}}

\[t = \frac{\bar{X_1}-\bar{X_2}}{s\sqrt{(\frac{1}{n_1}+\frac{1}{n_2})}}, \sim t_{n_1+n_2-2} under H_0\]

\[s^2 = \frac{(n_1-1)s_1^2+(n_2-1)s_2^2}{n_1+n_2-2}\]

\begin{Shaded}
\begin{Highlighting}[]
\NormalTok{s\_2 }\OtherTok{\textless{}{-}}\NormalTok{ (}\FunctionTok{var}\NormalTok{(wt1)}\SpecialCharTok{*}\NormalTok{(n1}\DecValTok{{-}1}\NormalTok{)}\SpecialCharTok{+}\FunctionTok{var}\NormalTok{(wt0)}\SpecialCharTok{*}\NormalTok{(n0}\DecValTok{{-}1}\NormalTok{))}\SpecialCharTok{/}\NormalTok{(n1}\SpecialCharTok{+}\NormalTok{n0}\DecValTok{{-}2}\NormalTok{)}
\NormalTok{s }\OtherTok{\textless{}{-}}\FunctionTok{sqrt}\NormalTok{(s\_2)}
\NormalTok{t }\OtherTok{\textless{}{-}}\NormalTok{ (}\FunctionTok{mean}\NormalTok{(wt1) }\SpecialCharTok{{-}} \FunctionTok{mean}\NormalTok{(wt0) )}\SpecialCharTok{/}\NormalTok{(s}\SpecialCharTok{*}\FunctionTok{sqrt}\NormalTok{(}\DecValTok{1}\SpecialCharTok{/}\NormalTok{n1}\SpecialCharTok{+}\DecValTok{1}\SpecialCharTok{/}\NormalTok{n0))}
\NormalTok{t0}\OtherTok{\textless{}{-}}\FunctionTok{qt}\NormalTok{(}\FloatTok{0.95}\NormalTok{,n1}\SpecialCharTok{+}\NormalTok{n0}\DecValTok{{-}2}\NormalTok{)}
\FunctionTok{abs}\NormalTok{(t);t0}
\end{Highlighting}
\end{Shaded}

\begin{verbatim}
[1] 0.6047303
\end{verbatim}

\begin{verbatim}
[1] 1.653043
\end{verbatim}

\[\because |t|=0.60 \le t_{n_1+n_2-2,1-\alpha/2} = 1.65,\alpha = 0.10\]

So, we cannot reject the null hypothesis that the hypothesis that both
groups have the same mean.

\hypertarget{problem-3}{%
\section{Problem 3}\label{problem-3}}

According to the CDC, approximately 20\% of pregnant American women
suffer from hypertension. Do our data support this claim? (Use column ht
- a value of 1 means the mother is suffering from hypertension.)

\hypertarget{a-conduct-a-99-confidence-interval-and-interpret-the-results.-what-can-we-conclude-about-the-cdcs-claim-from-this-interval}{%
\paragraph{a) Conduct a 99\% confidence interval and interpret the
results. What can we conclude about the CDC's claim from this
interval?}\label{a-conduct-a-99-confidence-interval-and-interpret-the-results.-what-can-we-conclude-about-the-cdcs-claim-from-this-interval}}

\hypertarget{b-conduct-a-one-sided-hypothesis-test-at-the-ux1d6fc-0.1-level.-in-this-test-we-want-to-see-if-the-true-proportion-is-indeed-less-than-the-claimed-20.-what-can-we-conclude-about-the-cdcs-claim}{%
\paragraph{b) Conduct a one-sided hypothesis test at the 𝛼 = 0.1 level.
In this test, we want to see if the true proportion is indeed less than
the claimed 20\%. What can we conclude about the CDC's
claim?}\label{b-conduct-a-one-sided-hypothesis-test-at-the-ux1d6fc-0.1-level.-in-this-test-we-want-to-see-if-the-true-proportion-is-indeed-less-than-the-claimed-20.-what-can-we-conclude-about-the-cdcs-claim}}

\hypertarget{problem-4}{%
\section{Problem 4}\label{problem-4}}

\hypertarget{is-there-a-difference-between-uterine-irritability-in-the-group-of-pregnant-women-who-smoke-vs-the-group-of-pregnant-women-that-dont-smoke-use-columns-ui-and-smoke.}{%
\paragraph{Is there a difference between uterine irritability in the
group of pregnant women who smoke vs the group of pregnant women that
don't smoke? (Use columns ui and
smoke.)}\label{is-there-a-difference-between-uterine-irritability-in-the-group-of-pregnant-women-who-smoke-vs-the-group-of-pregnant-women-that-dont-smoke-use-columns-ui-and-smoke.}}

\hypertarget{conduct-a-hypothesis-test-at-the-ux1d6fc-0.01-level.-what-can-we-conclude-about-the-proportions-of-women-with-uterine-irritability-between-the-smoking-groups}{%
\paragraph{Conduct a hypothesis test at the 𝛼 = 0.01 level. What can we
conclude about the proportions of women with uterine irritability
between the smoking
groups?}\label{conduct-a-hypothesis-test-at-the-ux1d6fc-0.01-level.-what-can-we-conclude-about-the-proportions-of-women-with-uterine-irritability-between-the-smoking-groups}}

\hypertarget{problem-5}{%
\section{Problem 5}\label{problem-5}}

Is race related to birth weight? (Use columns race and bwt.)

\hypertarget{a-what-test-would-be-most-appropriate-to-answer-this-question}{%
\paragraph{a) What test would be most appropriate to answer this
question?}\label{a-what-test-would-be-most-appropriate-to-answer-this-question}}

\hypertarget{b-what-assumptions-are-we-making-by-using-this-test-are-all-assumptions-met}{%
\paragraph{b) What assumptions are we making by using this test? Are all
assumptions
met?}\label{b-what-assumptions-are-we-making-by-using-this-test-are-all-assumptions-met}}

\hypertarget{c-conduct-the-test-at-the-5-significance-level-and-interpret-your-results.-be-sure-to-write-the-hypothesis-you-are-testing.}{%
\paragraph{c) Conduct the test at the 5\% significance level and
interpret your results. Be sure to write the hypothesis you are
testing.}\label{c-conduct-the-test-at-the-5-significance-level-and-interpret-your-results.-be-sure-to-write-the-hypothesis-you-are-testing.}}

\hypertarget{d-perform-multiple-comparisons---which-races-are-significantly-different-interpret-your-results.}{%
\paragraph{d) Perform multiple comparisons - which races are
significantly different? Interpret your
results.}\label{d-perform-multiple-comparisons---which-races-are-significantly-different-interpret-your-results.}}

\end{document}
